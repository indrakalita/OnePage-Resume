\documentclass{resume} % Use the custom resume.cls style
\usepackage{color}
\usepackage{hologo}
\usepackage[inline]{enumitem}
\usepackage[empty]{fullpage}
\usepackage[left=0.2in,top=0.0in,right=0.2in,bottom=0.0in]{geometry} % Document margins
\usepackage{enumitem}
\usepackage{fontawesome5}
\usepackage{hyperref}
\begin{document}
	\begin{center}
		\textbf{\textsc{DR. INDRAJIT KALITA}} \\
		\textsc{Postdoctoral Researcher}, Computing and Data Sciences (CDS), Boston University \\
		\href{https://indrajitkalita.com/}{https://indrajitkalita.com/} \hspace{1em} | \hspace{1em} \faLinkedin \ \href{https://linkedin.com/in/kalita-indrajit/}{linkedin.com/in/kalita-indrajit/}  \hspace{1em} | \hspace{1em} \href{mailto:indrajit@bu.edu}{indrajit@bu.edu} \\
	\end{center}
\vspace{-5px}
\begin{rSection}{PROFESSIONAL APPOINTMENTS}
\vspace{-5px}
\begin{tabular}{p{\dimexpr\textwidth-4cm}r}
	\textbf{Postdoctoral Researcher}, Computing and Data Sciences (CDS), Boston University, Boston, USA & 2023 -- Present \\
	\textbf{Research Associate}, CYENS-Centre of Excellence (SuPerWorld MRG), Nicosia, Cyprus & 2022 -- 2023 \\
	\textbf{Assistant Project Engineer}, Indian Institute of Technology Guwahati, India & 2014 -- 2015 \\
\end{tabular}
\end{rSection}
\vspace{-5px}
\begin{rSection}{Teaching Experience}
	\vspace{-4px}
	\begin{tabular}{p{\dimexpr\textwidth-4cm}r}
		\textbf{Teaching Assistant}, Indian Institute of Information Technology Guwahati (IIITG) & 2017 -- 2022 \\
		\textit{Courses:} C Programming Lab, Data Structure Lab, Design and Analysis of Algorithm Lab, Database and Management System Lab, Machine Learning Lab, Deep Learning Lab
	\end{tabular}
\end{rSection}

\vspace{-5px}
\begin{rSection}{Education}
		\vspace{-4px}
\begin{tabular}{p{\dimexpr\textwidth-4cm}r}
	\textbf{PhD in Computer Science \& Engineering, Indian Institute of Information Technology Guwahati} & 2017 -- 2022 \\
	\textit{Dissertation Title:} “Deep learning based adaptive land cover monitoring by analyzing remotely sensed images”, awarded September 2022 & \\
	\textbf{Master of Technology in Information Technology, Tezpur University} & 2015 -- 2017 \\
	\textit{Dissertation Title:} “Image Classification using Deep Convolutional Neural Network”, CPI 9.06/10 with Distinction & \\
	\textbf{Bachelor of Engineering in Computer Science \& Engineering, Gauhati University} & 2009 -- 2013 \\
	Percentage: 71.72\%, First class & \\
\end{tabular}
\end{rSection}

\vspace{-5px}
\begin{rSection}{Research Role and Major Projects}
	\vspace{-4px}
\begin{tabular}{p{\dimexpr\textwidth-4cm}r}
	\textbf{Project 1:} Rainfall Prediction in West Africa Using Deep Learning & 2023 -- Present \\
	\textbf{Project 2:} Land Use and Land Cover (LULC) Mapping and Environmental Monitoring in Cyprus & 2022 -- 2023 \\
	\textbf{Project 3:} Cyprus TreeMapper - Tree Detection and Counting & \\
	\textbf{Project 4:} Development of GAEA - Advanced Geo-Analytical Tool in Cyprus & \\
	\textbf{Project 5:} Innovative Integrated Tools and Technologies to Protect and Treat Drinking Water from Disinfection Byproducts (DBPs) (Project H2OforAll) & \\
\end{tabular}
\end{rSection}
\vspace{-5px}
\begin{rSection}{Research Expertise}	
		\vspace{-4px}
	\begin{tabular}{p{18cm}}
		\textbf{Deep Learning:} Optimization, CNNs, LSTMs, GNNs, Autoencoders for various applications \\
		\textbf{Computer Vision:} Image classification/segmentation, change detection, super-resolution \\
		\textbf{Programming: } C, C++, Python, PyTorch, Tensorflow (keras), MATLAB \\		
	\end{tabular}
\vspace{-5px}	
\end{rSection}

\begin{rSection}{SELECTED PUBLICATIONS (2 OF 18)}
\vspace{-4px}
\begin{enumerate}
 \item \textbf{Kalita, I.}, \& Roy, M. (2022). Class-Wise Subspace Alignment-Based Unsupervised Adaptive Land Cover Classification in Scene-Level Using Deep Siamese Network. \textit{IEEE Transactions on Neural Networks and Learning Systems}, 34(7), 3323-3334.
 \vspace{-8px}
\item \textbf{Kalita, I.,} \& Roy, M. (2020). Deep neural network-based heterogeneous domain adaptation using ensemble decision making in land cover classification. \textit{IEEE Transactions on Artificial Intelligence}, 1(2), 167-180.
\end{enumerate}
\end{rSection}
\vspace{-5px}	
\begin{rSection}{ACADEMIC AWARDS}
		\vspace{-4px}
	\begin{tabular}{p{16cm}r}
		Year-Round Internship, CYENS-Centre of Excellence, Cyprus & 2020 \\
		Top 200 in SAMADHAN Online Challenge, Ministry of Human Resource Development, India & 2021 \\
		Top 10 in Machine Learning Feature Extraction Competition, ICETCI \& NRSC-ISRO & 2021 \\
		First Runner-up, AAROHAN-2013 State-Level Project Competition, ABVP, India & 2013 \\
	\end{tabular}
\end{rSection}

\vspace{-5px}
\begin{rSection}{Professional Affiliations \&VOLUNTEER Activities}
		\vspace{-4px}
	\textsc{\textbf{Professional Memberships:}}  IEEE Member (ID: 93680859)\\
	\textbf{Journal Reviewer:} IEEE-TAI, T\&F-IJDE, Elsevier-ASR, Springer-Discover Computing\\
	\textsc{\textbf{Administrative Activities:}} 2021-2022: General Secretary of the Welfare Board, IIIT Guwahati
\end{rSection}
\end{document}
